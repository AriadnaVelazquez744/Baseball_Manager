\documentclass{report}
\usepackage[spanish]{babel}
\usepackage[left=2.5cm, right=2.5cm, top=3cm, bottom=3cm]{geometry}
\usepackage{enumerate}
\usepackage{graphicx}
\usepackage{booktabs}
\usepackage{tabularx}
\usepackage{enumitem}
\usepackage{amsmath}
\usepackage{amsfonts}
\usepackage{float}
\usepackage{hyperref}   

\setlength{\parindent}{0pt}

\begin{document}

    \begin{titlepage}
        \centering
        {\bfseries\Huge Manual de Usuario \par}
        \vspace*{1cm}
        \vspace*{3cm}
        \vspace*{1cm}
        {\LARGE \textbf{Nombre: Gestión de campeonatos de béisbol} }
        \vfill
        {\bfseries\LARGE Autores: \par}
        {\Large Ariadna Vel\'azquez Rey  C311 \par} 
        {\Large L\'ia L\'opez Rosales  C312 \par} 
        {\Large Carlos Daniel Largacha Leal  C312 \par} 
        {\Large Gabriel Andr\'es Pla Lasa  C311 \par} 
        {\Large Raidel Miguel Cabellud Lizaso C311 \par} 
        \vfill
    \end{titlepage}

    \section*{Objetivo del proyecto}
    Este proyecto tiene como objetivo el desarrollo de una aplicación web que facilite la gestión de los 
    campeonatos de béisbol. Su propósito principal es permitir a los usuarios consultar y analizar información 
    relevante de las series y los peloteros, proporcionando herramientas que soporten la toma de decisiones 
    basada en estadísticas detalladas y actualizadas. El sistema será accesible, eficiente y adaptado a las 
    necesidades de diversos usuarios, desde administradores hasta periodistas.


    \section*{Requerimientos Técnicos}
    \begin{itemize}
        \item Dispositivos capaces de ejecutar navegadores modernos y conexiones a internet estables.
        \item En caso de tener una versión local también son necesarios más de 300 mb disponibles para la instalación.
    \end{itemize}
    
    \section*{Requerimientos de Software}
    Para ejecutarla de forma local es necesario seguir las siguientes orientaciones: 
    \begin{itemize}
        \item Tener instalado los paquetes para manejo de PostgreSQL, crear una base de datos de tipo psql y editar 
        los datos de especificación en el archivo .env de la carpeta principal.
        \item Tener instalado el compilador de Python y las librerías que aparecen en el archivo principal 
        pyproject.toml.
        \item Instalar con 'npm install' desde la dirección de archivo Baseball\_Management/src los módulos para 
        ejecutar React.
    \end{itemize}

    \section*{Forma de instalar la aplicación}
    Para ejecutarla desde una versión local hay dos maneras de hacerlo:
    \begin{enumerate}
        \item Ejecutar en la consola (con el permiso -x) el archivo run.sh.
        \item Ejecutar en la consola desde la dirección de la carpeta Baseball\_Management el comando 'npm start'.
    \end{enumerate}

    \section*{Breve explicación de cada una de las opciones del sistema}

    \subsection*{Inicio de Sesión}
    Para acceder al sistema, los usuarios deben iniciar sesión con sus credenciales.

    \begin{figure}[H]
        \centering
        \includegraphics[width=0.8\textwidth]{img/img_1.png}
        \caption{Pantalla de inicio de sesión}
    \end{figure}

    
    \subsection*{Barra lateral en dependencia del rol}
    En dependencia del rol del usuario autentificado, en el barra lateral se mostrarán diferentes tipos de vistas.

    \begin{figure}[H]
        \centering
        \includegraphics[width=0.8\textwidth]{img/img_2}
        \caption{Barra lateral para Administrador}
    \end{figure}

    \begin{figure}[H]
        \centering
        \includegraphics[width=0.8\textwidth]{img/img_3}
        \caption{Barra lateral para el Director Técnico}
    \end{figure}

    \subsection*{Operaciones que puede realizar un Administrador}
    Los administradores tienen acceso a una vista `Formularios' en la cual pueden acceder a todas las tablas de la Base de Datos.

    \begin{figure}[H]
        \centering
        \includegraphics[width=0.8\textwidth]{img/img_4}
        \caption{Vista de Formularios}
    \end{figure}

    Los administradores pueden realizar cualquier tipo de operaciones CRUD (Create [Crear], Read [Leer], Update [Actualizar], Delete [Borrar]) sobre todas las tablas.

    \begin{figure}[H]
        \centering
        \includegraphics[width=0.8\textwidth]{img/img_5}
        \caption{Leer las filas de una tabla}
    \end{figure}

    \begin{figure}[H]
        \centering
        \includegraphics[width=0.8\textwidth]{img/img_6}
        \caption{Crear una nueva fila}
    \end{figure}

    \begin{figure}[H]
        \centering
        \includegraphics[width=0.8\textwidth]{img/img_7}
        \caption{Modificar una fila existente}
    \end{figure}

    \begin{figure}[H]
        \centering
        \includegraphics[width=0.8\textwidth]{img/img_8}
        \caption{Eliminar una fila}
    \end{figure}

    \subsection*{Operaciones que puede realizar un Director Técnico}
    Un Director Técnico puede ver los diferentes cambios que se han realizado a las alineaciones que posee su equipo.

    \begin{figure}[H]
        \centering
        \includegraphics[width=0.8\textwidth]{img/img_9}
        \caption{Cambios en las alineaciones de un Equipo en el campo}
    \end{figure}

    Un Director Técnico puede efectuar un cambio entre dos jugadores de su equipo que ocurrió en un determinado juego.
    
    \begin{figure}[H]
        \centering
        \includegraphics[width=0.8\textwidth]{img/img_10}
        \caption{Cambio de jugadores}
    \end{figure}


    \subsection*{Operaciones que puede realizar cualquier tipo de usuario}

    \subsubsection*{Consultas}
    Un usuario, sin importar el rol (en los que se incluyen los Usuarios Generales), puede realizar diferentes consultas sobre datos, como las series, equipos, peloteros o cualquier otra de información disponibles.

    \begin{figure}[H]
        \centering
        \includegraphics[width=0.8\textwidth]{img/img_11}
        \caption{Vista de las Consultas}
    \end{figure}

    En una consulta un usuario puede realizar diferentes operaciones seleccionadas, como listar y filtrar por nombre, número o fecha.

    \begin{figure}[H]
        \centering
        \includegraphics[width=0.8\textwidth]{img/img_12}
        \caption{Listar una tabla de la consulta}
    \end{figure}

    \begin{figure}[H]
        \centering
        \includegraphics[width=0.8\textwidth]{img/img_13.png}
        \caption{Filtrar por nombre una columna de una consulta}
    \end{figure}

    \begin{figure}[H]
        \centering
        \includegraphics[width=0.8\textwidth]{img/img_14.png}
        \caption{Filtrar por número una columna de una consulta}
    \end{figure}

    \begin{figure}[H]
        \centering
        \includegraphics[width=0.8\textwidth]{img/img_15.png}
        \caption{Filtrar por fecha una columna de una consulta}
    \end{figure}

    \subsection{Estadísticas}
    Un usuario cualquiera tiene acceso a diferentes reportes seleccionados sobre las estadísticas de los Campeonatos de Béisbol.

    \begin{figure}[H]
        \centering
        \includegraphics[width=0.8\textwidth]{img/img_16}
        \caption{Vista de las Estadísticas}
    \end{figure}

    Un usuario puede filtrar un reporte por un campo seleccionado.
    \begin{figure}[H]
        \centering
        \includegraphics[width=0.8\textwidth]{img/img_17}
        \caption{Filtrar campos en los Reportes}
    \end{figure}


    \begin{figure}[H]
        \centering
        \includegraphics[width=0.8\textwidth]{img/img_17}
        \caption{Reporte exportado en formato PDF}
    \end{figure}

    \section*{Breve explicación de cada una de las salidas del sistema}

    Un usuario puede exportar uno o varios reportes a determinados formatos (PDF o CSV, ya predeterminados).

    \subsection*{Reporte de Lanzadores}
    Este reporte muestra el total de juegos ganados y el promedio de carreras limpias permitidas por un lanzador.

    \begin{figure}[H]
        \centering
        \includegraphics[width=0.8\textwidth]{img/img_18}
        \caption{Reporte de Lanzadores}
    \end{figure}

\end{document}